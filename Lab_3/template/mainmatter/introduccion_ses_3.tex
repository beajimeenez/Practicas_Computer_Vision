\chapter{Sesión 3: Procesamiento de imágenes y extracción de características}
\label{chapter:introduction_ses_3}

\section{Materiales}

En esta práctica se trabajará con los siguientes recursos (puede encontrarlos en la sección de Moodle \textit{Laboratorio/Sesión 3}):

\begin{itemize}
    \item \textbf{\texttt{partA-B\_to-do.ipynb}}: notebook con el código que deberá desarrollar en los apartados A y B.
    \item \textbf{\texttt{partC\_to-do.ipynb}}: notebook con el código que deberá desarrollar en el apartado C.
    \item \textbf{helpers}: Archivos con métodos que le ayudarán a obtener los resultados. 
    \begin{itemize}
        \item \texttt{utils.py}
    \end{itemize}
    \item \textbf{data}: carpeta con imágenes para trabajar durante la práctica.
\end{itemize}

\section{Apartados de la práctica}

La Sesión 3 del laboratorio está dividida en los siguientes apartados:

\begin{itemize}
    \item Librerías: Importación de las librerías que se utilizan en la sesión. Se recomienda realizar la importación en una celda inicial para mantener la organización del Notebook.
    \item Apartado A: Filtro Gaussiano y Detección de bordes.
    \item Apartado B: Operadores Morfológicos.
    \item Apartado C: Detección de esquinas.
\end{itemize}

\section{Observaciones}

Aunque el guion de la práctica y los comentarios en Markdown del Notebook estén escritos en español, observe que todo aquello que aparece en las celdas de código está escrito en inglés. Es una buena práctica que todo su código esté escrito en inglés.

Aquellas partes del código que deberá completar están marcadas con la etiqueta \textbf{\texttt{TODO}}.

Es muy importante que trabaje consultando la documentación de OpenCV\footnote{\href{https://docs.opencv.org/4.x/index.html}{Documentación de OpenCV}: \url{https://docs.opencv.org/4.x/index.html}} para familiarizarse de cara al examen. Tenga en cuenta que en los exámenes no podrá utilizar herramientas de ayuda como Copilot.

\section{Qué va a aprender}

Al finalizar esta práctica, implementará filtros Gaussianos, y aplicará métodos de detección de bordes como Sobel y Canny. Además, comparará sus implementaciones con las funciones predefinidas en skimage. También utilizará operadores morfológicos y comprenderá su utilidad. Además sabrá como utilizar los detectores de esquinas.

\section{Evaluación}

La nota que obtenga en esta sesión de laboratorio será la misma que para ambos miembros de la pareja. Los apartados de la práctica serán evaluados como refleja la Tabla \ref{table:evaluacion}.

\begin{table}[h!]
    \centering
    \begin{tabular}{|c|c|c|}
    \hline
    \textbf{Tarea} & \textbf{Valor} & \textbf{Resultado} \\
    \hline
    Pregunta A.1 & 2.0 & \\
    \hline
    Pregunta A.2 & 2.0 & \\
    \hline
    Pregunta B.1 & 2.0 & \\
    \hline
    Pregunta C.1 & 2.0 & \\
    \hline
    Pregunta C.2 & 2.0& \\
    \hline
    \textbf{Total} & \textbf{10.0} & \\
    \hline
    \end{tabular}
    \caption{Valoración de los apartados de la práctica.}
    \label{table:evaluacion}
\end{table}
