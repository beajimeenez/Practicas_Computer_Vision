\chapter{Apartado B: \textbf{Operadores morfológicos}}
\label{chapter:tarea_b}

\section*{Tarea B.1: Implementación de operadores morfológicos}
\phantomsection
\addcontentsline{toc}{section}{Tarea B.1: Implementación de operadores morfológicos}

En esta sección debe implementar los operadores morfologicos que se indican a continuación.

\section*{Preguntas}
\addcontentsline{toc}{section}{Preguntas}

\vspace{5mm}
\begin{tcolorbox}[colback=gray!10, colframe=gray!30, coltitle=black, title=Pregunta B.1, halign=left]
Implemente los operadores morfológicos dilatación y erosión. Utilice para ello un kernel de 3x3 con valores 255 (blanco) como structuring element. No olvide binarizar su imagen antes de aplicar los operadores morfológicos.
\end{tcolorbox}